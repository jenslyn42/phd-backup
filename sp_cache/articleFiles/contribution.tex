\section{Baseline Competitors}

Intro to section; Why do we have baseline competitors, which ones will be introduced, and why those? 

LRU is a competitor, but can not achieve good performance because:
\begin{itemize}	
	\item Has no way to determine the usefulness of adding a path (i.e. no scoring function)
	\item Even if a path P is good (covers many queries), then if a sequence of consequtive queries comes which P can not cover, then it will be evicted.
	\item Has no way to optimize the number of paths in the cache, so available cache space may go unused.
	\item Querying the cache may require a scan of all paths in the cache
\end{itemize}

\section{Contribution} \label{sec:contribution}

Restate goal 1 \& 2. Overall divide into 2 sections, each focused on the methods solving goal 1 and 2 respectively. \\

Benefit equation.\\

\subsection{partition map} 
solves goal 1. reduce time on SP calc as it aids the cache with information on which paths may be useful.\\ 


\subsection{static cache} 
solves goal 2. has zero maintenance cost after filling the cache.


\subsection{cache representations and cache concepts:} 

\subsection{simple array of paths} - baseline for goal 2, unoptimized and expensive to use.

\subsection{simple array of paths inverted list} - solves goal 2, reduces the query time of the cache.

\subsection{graph representation} - solves goal 1, allows for more paths in cache, which should translate more cache hits.

\subsection{Sharing subpaths} - solves goal 1, allows for more paths in cache, translating to more cachehits. Unfortunally has a negative impact on goal 2 as it introduces some overhead in query time.


\subsection{Greedy algorithm}
shows in more detail how we propose to solve our problem
