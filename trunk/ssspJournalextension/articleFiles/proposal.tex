\section{Proposal of project parts}

% Mostly want to focus on the anonymization of GPS traces. I am having great difficulties coming up with a god way
% of doing this though. Not what I want the end result to be like, but more solution-wise how to archeve this in
% an efficient and nice way that I don't feel is just incremental compared to previous work.\\
% This is where I could use some suggestions on papers to read, keywords to think about, or maybe I could "just" take an
% existing suppression approach and use it with my concepts of time and sensitivity levels (see bullets below).
% 
% Remember I would like this to at some point be able to be published, so I want something that I can do alone, but
% I need help defining the scope of the project.
% I need your help on how to find the balance, so I can hopefully archeve this goal of a publication in the end of next
% this or next semester.
% (maybe suggestions on how to split the work, so I have a good project, but only "half" paper).\\

In the last meeting we talked about expanding, and classifying, the motivational scenarios, especially with regards to 
finding applications which has emphasis on working on subtrajectories. I have however been unable to come up with any good
scenarios, since I feel I am constantly ending up with more "density query" scenarios, which is not the kind of problem I want
to work on (which is why I have not added more scenarios in that category). 

Although I feel I have come a step closer to coming up with a solution for my proposed problem, my problem is still that I
feel like anything I come up with is too similar to existing work.

The "Preliminary Problem definition" more or less contains duplicate definitions for all concepts ("intro" \& subsections), this is on purpose, so I am
certain that I define all the concepts that I use so far. 

I am still a bit uncertain as to how much I should try to define in terms of classes/sets/tuples.



\subsection{Proposed system features}
\begin{itemize}
	\item Work on historical data (+ maybe updating the data continuesly with live data, depending on final solution).
	\item handle sensitivity groups for users with time (different groups sensitive at different times).
Groups are ways for "the system" classify users based on how much privacy they want in a given area, at a specific
time range. Privacy is here the degree of anonymization a user wants guarantied to be willing to publish his data.
	\item Have possibility to weight what is most sensitive factor when trying to anonymize: temporal or spacial.
Taking the two extremes: only using temporal or spacial anonymization, they have the following impact:
	\begin{itemize}
		\item Using the spacial dimension as the important factor in anonymization the location of the users will be generalized
uptil a factor satisfying the users privacy settings (street, block, part of city, etc.)
		\item When considering the temporal dimension as the important factor in anonymization, the user will have his trajectory
generalized over a time interval, making it impossible to identify when exactly the user traveled on the trajectory.

The user may not have a problem with someone figuring out where is was at a single point in time, 
but he may not want someone to find out that he is taking the same route at different times every day.
	\end{itemize}
\end{itemize}



\subsection{Preliminary Problem definition}

In this section we introduce relevant notations, define system requirements and specify
the behavior of the proposed service. 

We assume a setting where all users are equipped with a \ac{MD} able to communicate and report
the users position. All  \ac{MD}s are online and are continuesly reporting the users location at
predefined intervals. We use the terms {\it user, mobile device, and client} interchangeable and 
denote the set of \ac{MD}s by  $\mathbf{U} \subset \mathbb{N}$.

Let us assume a 2D scenario, where the movements of users in $\mathbf{U}$ are restricted to a road network 
$G(V,E)$. $V$ is the set of vertices, where each one represents either a street intersection or an important 
landmark. $E$ is the set of directed edges, where each one represents the smallest unit of road segment. All
users $u \in \mathbf{U}$ are mapped to edges in $e \in \mathbf{E}$.

All users specify at least one privacy profile. Each privacy profile specify the privacy settings for each road segment $e$
such that for each road segment $e_i \in \mathbf{E}$ that the user traverses, the system will generalize $e_i$ in 
the spatio-temporal domain



\subsubsection{Users}
Users are represented by a set $U$ where each user $u \in U$

\subsubsection{Road network}
The approach will assume all users $u \in U$, move on a road network which we define as a graph $G(V,E)$. $V$ is the set
of vertices, where each one represents either a street intersection or an important landmark. $E$ is the
set of directed edges, where each one represents the smallest unit of road segment. 

\subsubsection{Trajectories}
As the lowest level, a trajectory is a set of tubles $(time,longitude,latitude)$ ordered by the time attribute. 
However, as the approach we want to end up with is going to work on a road network, a 
higher level of abstraction is more appropriate. We define $T$ as the set of trajectories, where each trajectory 
consist of an id$(tid)$ and sequence of edges traversed $(tid, \langle e_0,\ldots,e_j \rangle)$, where $e_i \in E$,
the set of edges. I assume trajectories only move on the road network, and users users must traverse the entirety
of an edge. 

\subsubsection{Historical data}
The historical data consist of $T$, sorted by start/end time. This enables the approach to use
an appropriate time slice $(tid_g,\ldots,tid_h)$, where $tid_i \in T$, to do spatial anonymization, or work on all of $T$
to do temporal anonymization


\subsubsection{Privacy settings}
For the approach to use, for the user, satisfactory levels of anonymization, the user needs to be 
able to specify how $sensitive$ different areas(auto mapped to the edges $e \mid e \in E$ included in the area) are to the
user at different times in the day. Users will also be put into groups



\subsection{Taxonomy on Spatial-Temporal}
The proposed solution works on the spatio-temporal domain. 
Since a place/object can have different degrees of sensitivity associated with it at different 
times it is important to consider both the spatial and the temporal domain when trying to ensure 
the privacy of users in the solution.



\subsection{Privacy Profile}

Privacy profiles are a class of time constrained objects specifying a users road segment constrained
spacial sensitivity for discrete time intervals.


\subsubsection{list of things user would like to specify}

\begin{itemize}
	\item Sensitive places / road segments
	\begin{itemize}
		\item various degrees of sensitivity, some road segments/locations may
		be more sensitive than others e.g. a user may find it annoying if
		people knew he was shopping in some fancy store where things are expensive,
		but only annoying, while he may be really upset if someone found out he
		was seeing a psychiatrist. Thus the two have different levels of sensitivity. 
		\item different sensitivity degree as different points in time. e.g. 
		the free clinic is not a sensitive point on the trajectory at 8 am or 4 Pm when
		the user is driving too and from work (assuming of cause it is on the path), but any
		other time it would be a highly sensitive location/area.
	\end{itemize}
	\item Different profiles, e.g. a work profile, vacation profile
	\begin{itemize}
		\item profiles could be activated with a time limitation, he could have a choice of doing 
		it either automatically or manually.
	\end{itemize}
	\item Minimum privacy 
	\begin{itemize}
		\item can be specified globally for any trajectory, or for each level of sensitivity
		according to the area (road segments) the user is passing through.
	\end{itemize}
	\item 
\end{itemize}

\subsection{Motivational uses}
For the users publishing their data, 
they might want to access the trajectory dataset (possibly through a third party service provider on the data) to e.g. enable their 
GPS device to make better route planning depending on the time and place (some GPS devices already have the ability to use 
semi-live data to be better at suggesting routes to the user).

The user may also use the data through a service provider to e.g. find car pooling opportunities to an from work based
on what other users also takes the (close to) same route within a user specified time interval.

The users might also just get nothing in return for publishing their trajectories, but might help the government (or
similar organization) to better plan and build roads in the future. For some people the sense of "importance" may 
be motivational enough to make them publish their data.



\subsection{Scenarios}
\begin{itemize}
	\item {\bf Premise}
	\begin{itemize}
		\item {\bf User} Users are willing to publish their trajectories, if they are guarantied that no sensitive
or personally identifying information is contained or deduceable from the data set.
		\item {\bf Company} Companies wants to be able to query a public database with trajectories. They want the data
to be as correct, complete and detailed as possible
		\item {\bf service provider:} Collects, maintains and anonymizes a database of trajectories. Will make this dataset 
available to Companies.
	\end{itemize}
	\item {\bf Density queries}
	\begin{itemize}
		\item {\bf Scenario One} A company wants to monitor when, and where, certain road segments are congested in order to
{[optimize the duration of red/yellow/green at certain points in time, to make traffic flow better],
[Advise the city about future investments in roads and public transport],
[devise a system (for e.g. GPS devices) which is always able to predict the fastest route from point A to point B, irrespective of the actual length of the suggested route]}
and thus make some money.
		\item {\bf Scenario Two} A city or government could use collection of trajectories to analyze where people are going 
to and from at different time intervals during the day, or even during the calendar year. This could be usefull to gain an 
understanding of the general traffic patterns on their road network. This information could be usefull at several granularities 
to plan investments in new roads, how often existing roads should be serviced, or just to devise an automatic road congestion 
warning system
	\end{itemize}
	\item {\bf Outlier detection}
	\begin{itemize}
		\item {\bf Scenario Three} The police could do analysis of speed outliers on the dataset to see which road segments people are often driving to fast, 
and then set up speed controls on those road segments more often. They could also look at the opposite, people stopping, on major roads, 
where people would not normally have reason to stop, this could indicate traffic accidents, which could prompt more police presence on those roads
to make people drive nicer.
		\item {\bf Scenario Four} It could be usefull to do trajectory similarity analysis on many different levels. If is was possible
for e.g. city officials to specify two points (or spatial areas) A/B they could do some analysis on how people who go from A to B travel i.e.
what routes/trajectories they take to travel from A to B. It could also be examined if pattern changes over a day/week/month/year.
This kind of analysis could e.g. be used to calculate the inter-dependency between roads, which could be used to predict the increase in
load on other roads if one road is closed for some reason (e.g. maintenance, accident, or obstruction from some activity like a caneval or the like)
		%\item {\bf Scenario Five} stuff...
	\end{itemize}
	
\end{itemize}