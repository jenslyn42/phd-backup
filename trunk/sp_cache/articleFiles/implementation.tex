\section{Experiments}

Introduce what data is used and how we generate synthetic data \\
Introduce standard test parameters for the tests to follow.\\
\begin{tabular}{|l |l |l|}
\hline
\textbf{Parameter} & \textbf{Meaning / used for} & \textbf{Standard value} \\\hline
Mapfile & The map which the test is performed on & \\\hline
NumQueries & Number of \spath queries in test & \\\hline
QuerySet & which dataset is used to provide queries & \\\hline
TrainSet & For generating region statistics, which dataset is used & \\\hline
CacheSize & Size of cache in bits & \\\hline
cacheType & Type of cache representation (list/graph) & \\\hline
kD-tree & Hight of the kD-tree & \\\hline
\end{tabular}

Write experiments to examine performance of goal 1 \& 2 
Test ideas (several ideas may be combined, like item 1 can be done on all datasets from item 2):\\
\begin{itemize}
\item increase kd-tree hight from 0-10
\item different maps
\item compare cache type performance
\item compare with baseline methods.
\item vary the cache size
\item vary number of queries.
\end{itemize}