\section{Problem Definition}\label{sec:problemdef}

In this section we introduce relevant notations, formally define privacy
requirements and the behavior of our proximity based service.

We assume a setting where a set of mobile users form a social network. Every user
uses a mobile device(\mdns) with positioning and communication capabilities, allowing 
them to to be online, have access to a central location server (LS), and determine their own 
location. The terms \textit{mobile device}, \textit{user}, and \textit{client} 
are used interchangeably, denoting the set of all \mds by $\mathbf{M} \subset \mathbb{N}$. 
The social network containing users from $\mathbf{M}$ is defined by the friend-ship 
relation $\mathbf{F}$, where $\{(u,v), (v,u)\} \subseteq \mathbf{F}$ if $u,v \in \mathbf{M}$ 
are friends. 

Let us assume a 2D scenario, where users from $\mathbf{M}$ can freely move in
Euclidean space and every user $u \in \mathbf{M}$ at the current time
defines $loc(u)$ and $vic(u)$. Here $loc(u)=(loc(u).x, loc(u).y)$ represents
$u$'s 2D location and $vic(u)$ specifies $u$'s dynamic \textit{vicinity region}.

The vicinity region is a multi-parted
(composition of 1$\leq$ circle, polygon, etc.) region around a user
location and it can be understood as an infinite set of spatial points that
changes over the time. 


The privacy-aware proximity based service notifies user $u \in \mathbf{M}$ if
any of his friends $v \in \mathbf{M} | (u, v) \in \mathbf{F}$ enters his
vicinity region. More specifically, first all of $u$'s friends are classified to
be in proximity or not by checking following conditions:
\begin{enumerate}
    \item if $loc(v) \in vic(u)$ user $v$ is in $u$'s proximity;
    \item if $distLV(loc(v),vic(u)) > \lambda$, user $v$ is not in $u$'s
proximity;
    \item if $distLV(loc(v),vic(u)) \leq \lambda$, the service can freely choose
to classify $v$ as being in $u$'s proximity or not.
\end{enumerate}
Here $distLV(l,v)$ denote a shortest Euclidean distance between location $l$ and
vicinity region $v$. If $l$ is inside $v$ then $distLV(l,v) = 0$.The $\lambda
\geq 0$ is a service precision parameter and introduces a degree of freedom in
the detection of location-to-vicinity intersection. Note, that small values of
$\lambda$ corresponds to higher precision. When the classification is complete,
$u$ is provided with a set of proximate friends that now are classified as
being in proximity while they were not in proximity before.
Every user has to have ability to tweak their desirable service precision level
$\lambda$ and the service has to possibly minimize amount of client
communication depending on its $\lambda$ setting.

In addition to that, for every user $u \in \mathbf{M}$ the service has to
satisfy following location privacy requirements:
\begin{itemize}
 \item The exact location of $u$ is not disclosed to any party (e.g., any other
user or the LS).
 \item User $u$ allows nobody else but his friends to detect him in
their vicinities.
\end{itemize}

The following section details our proposed proximity based service that
meet these requirements.