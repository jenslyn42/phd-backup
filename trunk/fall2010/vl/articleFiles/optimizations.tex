\subsection{Incremental update optimization}\label{sec:optIncUpd}

According the protocol when one (or more) location or vicinity enclosing
granule changes due to user movement, user's encrypted data must
be updated on the \ls by sending a $M_{el}$ message. Thus, in most
cases user's two consecutive $M_{el}$ messages would contain duplicated 
encrypted granules. 

The communication can be reduced by enabling so called \textit{incremental
updates}(\iuns). Here we introduce new type of message $M_{elUpd}$ that clients 
are allowed to send, instead of the $M_{el}$, when their locations change. 
It contains old items $u$, $l$, $g^*_l$ from $M_{el}$ and new items 
$\mathbf{g^*_{vDel}}$, $\mathbf{g^*_{vIns}}$, where $\mathbf{g^*_{vDel}}$, 
$\mathbf{g^*_{vIns}}$ define encrypted granules that must be deleted and inserted 
on the \ls in order to fully update some user $u$ encrypted data for level $l$. 
More precisely, if $m_1$ and $m_2$ are two consequent messages of type 
$M_{el}$ such that $m_1.u$=$m_2.u$ and $m_1.l = m_2.l$ then the
client may send a message $m_3$ of type $M_{elUpd}$ instead of $m_2$, where
$m_3.\mathbf{g^*_{vDel}}$ = $m_1.\mathbf{g^*_{v}}$
$\backslash$ $m_2.\mathbf{g^*_{v}}$ and $m_3.\mathbf{g^*_{vIns}}$ =
$m_2.\mathbf{g^*_{v}}$ $\backslash$ $m_1.\mathbf{g^*_{v}}$. 
Figure \ref{fig:ui_rf}a visualizes locations, vicinities, and 
vicinity-intersecting granules of user $u$ at two consequent 
time steps 0 and 1. Darkened sets of cells $\mathbf{g_{vDel}}$ and 
$\mathbf{g_{vIns}}$ visualize unencrypted representation of sets $g^*_{vDel}$ 
and $g^*_{vIns}$. Note, that introduction of $m_3$ helps reducing communication
only if $|m_3.\mathbf{g^*_{vDel}}|$ + $|m_3.\mathbf{g^*_{vIns}}|$ $<$
$|m_2.\mathbf{g^*_{v}}|$. 


Our presented client and server algorithms (See Alg. \ref{algCl}, \ref{algSrv})
can be easily modified to support incremental updates. In the current
implementation once a user goes from higher to lower levels (switches into
coarser granularity) granules of active level are being removed from stacks
on client and server ($GS$ and $GL$) without possibility to reuse them. The idea
is to preserve these granules on both client and server such that it would be
possible to update them utilizing $\mathbf{g_{vDel}}$, $\mathbf{g_{vIns}}$ or
$\mathbf{g^*_{vDel}}$, $\mathbf{g^*_{vIns}}$. 

The incremental updates impact on client communication is evaluated in 
Sec. \ref{sec:performStudy}.

