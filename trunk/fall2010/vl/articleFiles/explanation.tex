\ffh{kind of long and verbose subsection...}
We the \ffns, developed in \cite{ffinder}
as a competing solution.
The \ff also utilizes the client-server paradigm and
it offers users location privacy via grid-based spacial
obfuscation techniques, and encryption.
The \ff offers equivalent notions of a maximum level and groups.
The \ff maps its location into its current grid column and row, as well
as the column and row respectively to the right and above the user. The
user then encrypts the two rows and columns for its current level, and send
the 4 and level to the server.
The \ff does proximity detection by letting the server detect if two users
share, at least, one row and column. (See fig. \ref{fig:expFFbehaviour}).
In figure \ref{fig:expFFbehaviour} the proximity detection by \ff
is illustrated. In \ref{fig:expFFbehaviour}b{\&}c the two proximities
possible is shown, in \ref{fig:expFFbehaviour}(b) two users detect proximity
by overlapping corner cells and in \ref{fig:expFFbehaviour}c the proximity is
detected by 4 overlapping cells of a side. in \ref{fig:expFFbehaviour}a u3 is 
not in proximity with anyone because his area of detection does not overlap with
anyone else.
\ffh{need more here} 


\begin{figure}
       \center
       \begin{tabular}{c c c c}
			    \includegraphics[width=0.33\textwidth]{images/proxDetDemo.pdf} &
			    \includegraphics[width=0.33\textwidth]{images/usrPosDemoA.pdf} & 
			    \includegraphics[width=0.33\textwidth]{images/usrPosDemoB.pdf} & 
			     \\
          (a) & (b) & (c)
       \end{tabular}
       \caption{\ff Behavior}
 \label{fig:expFFbehaviour}
\end{figure}


\subsection{Experimental Settings}
For each of our implementations we have build a
small parser, which in the case of the \vl is also
able to read the road network files of Oldenburg,
enableing roadnetwork constrictions in \vl.
%, as well as some posibilities for visualisation.
