\subsection{Grouping of Users}\label{subsec:grouping}

Currently all users in $\mathbf{M}$ share a single encryption function $\Psi$.
Security of $\Psi$ directly influence location privacy of all users in the
system. An adversary knowing $\Psi$ can easily decipher encrypted
granules of user current location and his vicinity. It is difficult to ensure
that the function $\Psi$ will stay secret in case of a high number of users
of the \vl service. Note that in case of leaked $\Psi$ users' minimum privacy 
requirements, specified by $L_{max}$, are still guaranteed, although it is 
desirable for users to obtain as much privacy as possible.

In order to limit affected users in case of leaked $\Psi$
\textit{grouping of users} can be enforced. Friend-grouping is
introduced in the long-paper version of the \ff \cite{ffinder}. The idea is
that all users in the system are grouped into possibly overlapping groups, so
that each user is put into one or more groups. Both friends and non-friends can
belong to the same group, but if two users are friends, they must be in at least
one common group. Each such group $G$ is assigned a distinct $\Psi_G$ function
and it is used by all members of $G$. Then if such $\Psi_G$ is leaked, only the
location privacy of the users in group $G$ are compromised.

Our presented algorithms of \vl can be easily modified to support friend
groups. The client and the server should treat each group individually such
that clients report their encrypted data for all groups that they are part of,
and the server independently analyzes encrypted data for every distinct group. 
In this paper, we do not consider how these groups are created. This can be done
automatically or manually by the users themselves.
