\section{Related Work} \label{sec:rel_work}
%
In this section we review general location privacy preserving techniques 
followed by the relevant work on location privacy in proximity detection
services.

\subsection{General Location Privacy Techniques}

In the most common setting assumed in location-privacy research,
an LBS server maintains a public set of points-of-interest (POI), e.g. gas
stations. The goal is then to retrieve from the server the nearest POIs to the
user, without revealing the user's location $q$ to the server. Many
location privacy solutions exist for this setting and they can be broadly
classified into two categories: spatial cloaking and transformation. 

Spatial cloaking \cite{gruteser03,mca06,duckham05a,acdvs07} techniques
generalize the user's exact location $q$ into a region $Q'$, which is then used
for querying the server. The region $Q'$ is then sent to the LBS server, which
returns all the results that are relevant to any point in $Q'$. Such technique
ensures that even if the attacker knows locations of all users, the identity of
the querying user can be inferred only with some probability. 

The transformation approaches \cite{ks07,ghinita08} map the user's location $q$
and all POIs to a transformed space, in which the LBS server evaluates queries
blindly without knowing how to decode the corresponding real locations of the
users.

In contrast, in the proximity detection problem, the users' locations are both
query locations and points-of-interest that must be kept secret. Thus the
existing spatial cloaking and transformation techniques, that assume public
datasets, cannot be directly applied for proximity detection. However the
concepts of spatial cloaking and transformation can be, and are used in the
existing privacy-aware proximity detection approaches.

\subsection{Privacy-aware proximity detection methods}

Ruppel et al. \cite{proxDetTrans} develop a centralized solution that supports
proximity detection with a certain level of privacy. It applies a distance-preserving 
mapping (a rotation followed by a translation) to convert the user's location $q$ into 
a transformed location $q'$. Then, a centralized proximity detection method is 
applied to detect the proximity among those transformed locations. However, Liu 
et al. \cite{lgk06} points out that such distance-preserving mapping is not safe 
and a attacker can easily derive the mapping function and compute the users' original 
locations. 

% maybe remove???
Mascetti et al. present a centralized solution
\texttt{Longitude} \cite{longitudePaper}.
Here users apply spatial cloaking followed by modular transformation prior
sending their 
locations the server. The applied transformation prevents the disclosure of 
location information, but introduces false proximity detections that do not occur in our 
proposed approach.


\hc, presented in \cite{pbsPaper}, is a privacy preserving solution which employ
the filter-and-refine paradigm. First, users spatially cloak their locations and sent them the
server. Then, the server computes a minimum and maximum distances (Fig. \ref{fig:relworkDemos}a)
between these cloaking regions. Depending on the specified thresholds and computed distances, 
the server classifies friends being in, not-in, or possibly-in proximity. The classification 
result is immediately reported to users, and in case of uncertainty, two users have to perform 
direct communication in order to refine their proximity status. 
In the refinement step first two users map their locations 
and vicinity regions into the some spatial subdivision, e.g. grid, and then they use secure 
two-party computation protocol to check if the mapped location of one user lays inside the 
mapped vicinity of other user. This can be performed with no need for the users 
to expose their mapped locations and vicinities. Depending on the result of the 
set-inclusion checking, either proximity or separation is detected. Figure
\ref{fig:relworkDemos}b visualizes a scenario, where user $u_2$'s grid-mapped location
(dark rectangle) intersect with $u_1$'s grid-mapped vicinity region. In this case, 
the proximity between $u_1$ and $u_2$ is detected.

Unlike in our approach, their proposal does not completely hide user locations
from the central server as it always knows their cloaked regions. If strong
privacy is required users are forced to perform user-to-user communication more
frequently thus significantly increasing the amount of client communication due to
an expensive secure two-party computation protocol. 



\begin{figure}
       \center
       \begin{tabular}{cc}
			   
\includegraphics[width=0.25\textwidth]{vl/images/pbsDemo.pdf} &
			   
\includegraphics[width=0.18\textwidth]{vl/images/pbsDemo2.pdf} \\
%	Filtering step & Refinement step \\
          (a) & (b) \\
\includegraphics[width=0.15\textwidth]{vl/images/proxDetDemo.pdf} &
{\scriptsize %\footnotesize
     \begin{tabular}[b]{| c | c |}\hline
        \textbf{Col/ Row} & \textbf{$\mathbf{\Psi}$} \\  \hline
        0 & $c_0$ (5)\\ \hline
        1 & $c_1$ (9)\\ \hline
        2 & $c_2$ (1)\\ \hline
        3 & $c_3$ (7)\\ \hline	
     \end{tabular}
} \\
          (c) & (d)
       \end{tabular}
       \caption{Behavior of \hc and \ff}
            \label{fig:relworkDemos}
\end{figure}



\v{S}ik\v{s}nys et al. \cite{ffinder} have developed a centralized privacy-aware
proximity detection method, called \ff, which provides strong privacy guaranties
and employs a grid-based technique to optimize communication cost. 
Users map their locations into 4 cells of the grid from a \textit{list of grids},
and, prior sending to the server, encrypt them using shared, one-to-one mapping encryption function.
The \textit{list of grids} globally defines a collection of grids, where cell sizes of distinct grids 
monotonically decreases with respect to the \textit{level number}. The server is able to 
perform a matching between encrypted coordinates of different users, for a shared
level, and report the matching result. When matchings at all consecutive levels 1
to $L_{max}$ are detected then actual proximity between two users is detected.
Pairs of users individually choose $L_{max}$ value which yields a
specific mutual vicinity radius for two users. The proximity is detected
iteratively such that after each encrypted coordinate match, 
the server checks if required level $L_{max}$ is reached. If so, then users are
informed about their proximity, otherwise the server sends a message to one or
both of the users asking them to use a finer grid, i.e., increase their current
level, for the next matching iteration. 

E.g. if the encryption function $\Psi$ is as defined in
\ref{fig:relworkDemos}d, then encrypted coordinates of users $u_1$, $u_2$, and
$u_3$ respectively are $(x:[c1,c2],y:[c0,c1])$, $(x:[c2,c3],y:[c1,c2])$,
$(x:[c0,c1],y:[c2,c3])$. Assuming that level of grid in Fig.
\ref{fig:relworkDemos}c is $L_{max}$, the user $u_1$ is in proximity with $u_2$,
however $u_3$ is not in proximity with either.



The limitation of the \ff is low, and uncontrollable, precision of the
proximity detection. On the proximity notification the actual distance between
two users can be any in the range from $d$ to $d + \lambda$, 
where $d$ is width/height of grid cell at level $L_{max}$, 
$\lambda = d(2\sqrt(2) - 1)$ is a service precision parameter. Note, 
that in most other proximity detection approaches, unlike in \ff, the parameter 
$\lambda$ can be chosen freely by users. 

None of the existing privacy-aware approaches directly support the ``river'' 
and the ``bar'' proximity detection scenarios (see Fig. 
\ref{fig:suppVic}c/\ref{fig:suppVic}d) since they use static circular-shape vicinity. 
However some solutions, like \hc \cite{pbsPaper}, might be adapted to support dynamic 
vicinities, though current limitations will persist and introduce new problems 
like e.g. how to efficiently update user locations and vicinities.




\subsection{Our contribution}

In this work we combine best features of the previously presented proximity
detection approaches to build our solution, the \vl. The solution combines the
ideas of encrypted coordinates, their blind matching at the server-side, and
the vicinity mapping into multi-spatial subdivisions, e.g. list of grids.
Our solution, unlike Mascetti et al. \cite{pbsPaper} proposal, employs only
centralized architecture, where the server knows no users' spatial data. It
allows users individually select preferable proximity detection precision and 
supports irregular-shape, changing over time vicinities. Our proposal
is will be targeted for 2D environment.
It can however easily be generalized to n-dimension.
