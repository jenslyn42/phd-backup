\section{Intro} % Bookmark information

\begin{frame}[red] %hmm.. thought i could change colour here :S
\frametitle{The Presentation}


\begin{itemize}
\item Not the usual slide style
\begin{itemize}\item More text - less code \end{itemize}
\item Interrupt if you are bored, and ask me something instead
\begin{itemize}\item E.g. "You are boring, show me how to code X" \end{itemize}
\end{itemize}


% What is GWT \\
% \indent	GWT vs. Competitors, GWT not just a AJAX library \\
% Widgets and composition \\
% RPC \\



% purpose - ease of coding \\
% 	- responsive UI \\
% 	- RPC to server side Java \\
% 	- Structure of project server/client code \\


\end{frame}


\subsection{GWT} % Bookmark information, displayed in the progress tree

\begin{frame}[red] %hmm.. thought i could change colour here :S
\frametitle{What is GWT - The Google Web Toolkit}

It is among other things:

\begin{itemize}
\item An AJAX framework
\item A Widget library
\item A Java-to-Javascript compiler
\end{itemize}

\end{frame}

\subsection{Motivation} % Bookmark information, displayed in the progress tree

\begin{frame}[red] %hmm.. thought i could change colour here :S
\frametitle{Motivation for using it}

\begin{itemize}
\item Communicate with your server through really simple RPC
\begin{itemize}\item  Using e.g. JSON and XML\end{itemize}

\item Optimize the JavaScript script downloads based on user profile
\begin{itemize}\item  Using Deferred binding\end{itemize}

\item Reuse UI components across projects

\item Use other JavaScript libraries and native JavaScript code
\begin{itemize}\item  Mix handwritten JavaScript in your Java code (JSNI)\end{itemize}

\item Easily support the browser's back button and history
\begin{itemize}\item  adding state to the browser's back button history \end{itemize}

\item Localize applications efficiently

\item Test your code with JUnit

\end{itemize}

\end{frame}


\subsection{Application Areas} % Bookmark information, displayed in the progress tree
\begin{frame}[red] %hmm.. thought i could change colour here :S
\frametitle{Application}

GWT is, unlike many other AJAX frameworks, aimed at both webpage- 
and webapplication- development.

\begin{itemize} 
 	\item The Javascript code is compiled to minimal size
 	\item The Javascript has any unreachable code purged at compile time
	\item Highly versatile and combinable UI widgets.
	\item Java source language makes it realistic to do actual application
	development in Javascript
\end{itemize}

\end{frame}
\begin{frame}[red] %hmm.. thought i could change colour here :S
\frametitle{Pros and Cons}

\begin{itemize} 
	\item Where to use it.
	\begin{itemize} 
	 	\item Webpages which needs to react to user actions immediately
	 	\item Anywhere you would want some dynamic content on your webpage
	\end{itemize}
	\item Where Not to use it.
	\begin{itemize} 
	 	\item Purely Static (informational) pages which does not change or update often.
	 	\item Anywhere it would be assumed users have disabled Javascript
	 \end{itemize}
\end{itemize}

\end{frame}

\subsection{Resources} % Bookmark information, displayed in the progress tree
\begin{frame}[red] %hmm.. thought i could change colour here :S
\frametitle{GWT Resources}

\begin{itemize}
\item Download GWT\\ http://code.google.com/webtoolkit/
\item Widgets Demo http://gwt.google.com/samples/KitchenSink/KitchenSink.html
\item Official Libraries http://code.google.com/webtoolkit/googleapilibraries.html
\end{itemize}
\end{frame}