\section{Experiments}
We perform experiments using two different datasets and compare our method to two baseline approaches. Our approach \osc is compared to \lru and \fifo

\ffh{intro of section and what will be presented, and in which order}

\ffh{briefly mention the implementation}



\subsection{Data Generation}

We generate queries with random start-/end-nodes in two ways: completely random and Gaussian distribution.


\ffh{Use Brinkhoff to generate realistic start-/end-pairs based on where people actually want to go (give OSC an advantage). Explain the Brinkhoff data generator, the facts about the Oldenburg map and the generator parameters}

The test program has been implemented in C++ and the tests have been run on a Core2 Duo@2.8GHz with 4 GB memory. Timings has been measured using normal clock time, with mapdata preloaded into memory.


\subsection{Test Parameters}


\begin{tabular}{|l | p{0.68\columnwidth}|}
\hline \textbf{Parameter} & \textbf{Purpose} \\\hline
numqueries & Number of queries used \\\hline
cacheSize & Number of nodes which can fit in the cache \\\hline
gaussian & Whether to use Gaussian distribution when generating queries \\\hline
useNodeScore & Use number of nodes in a cache element when calculating score in \osc \\\hline
useHitScore & Use number of cache hits of a cache element when calculating score in \osc \\\hline
double sigma & Sigma value when $gaussian$ is specified \\\hline
\end{tabular}



\subsection{Experiments}

\ffh{Write some test, generate graphs of them, explain graphs}

\ffh{explain why we have chosen the tests we have}
