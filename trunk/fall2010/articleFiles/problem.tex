\section{Problem setting}\label{sec:problemdef}

We introduce relevant notations, define system requirements and specify
the behavior of the proposed service. 


First we briefly summarize the general scenario again. We assume users are willing to upload the positioning data from the routes (trajectories) they take when they move around on a road network. The system then anonymizes the trajectories according to a user specified "Privacy Profile" (Sec.~\ref{sec:usrprofile}), specifying which places or routes are sensitive to the user (Fig.~\ref{fig:overview}A, 1-5). After anonymization the system makes the data available to potential service providers for whom it guaranties a minimum level of data quality.


\subsection{Problem Definition}\label{subsec:probdef}

\begin{table}
%\begin{tabular*}{0.8\columnwidth}{|p{0.2\columnwidth}|l|p{0.25\columnwidth}|}
\begin{tabular*}{\columnwidth}{|l||p{0.77\columnwidth}|}
\hline
Symbol		& Meaning \\\hline
MD		& Mobile Device \\\hline
$\mathbf{U}$	& Set of MDs \\\hline
$\mathbf{V}$	& Set of vertices \\\hline
$\mathbf{E}$	& Set of edges \\\hline
$G(\mathbf{V,E})$ & Road network \\\hline
$\mathbf{P}$	& Collection of all PSRs \\\hline
$\mathbf{T}$	& Set of trajectories \\\hline
$\Gamma$	& Set of all trajectories which contains sensitive PSR $p$ \\\hline
RT		& Road Type \\\hline
AS		& Protection scheme: Always Sensitive \\\hline
ASTI		& Protection scheme: Always Sensitive with time interval \\\hline
RS		& Protection scheme: Rarely sensitive \\\hline

\end{tabular*}
\caption{Table of symbols and notation} 
\label{tab:symbols}
\end{table}


We assume a setting where all users are equipped with a \md able to communicate and report
the users position. All  \mds are online and are continuesly reporting the users location at
predefined intervals. We use the terms {\it user, mobile device, and client} interchangeable and 
denote the set of \mds by  $\mathbf{U} \subset \mathbb{N}$. We expect a \md to be cabable of visualizing its current location.

We assume a 2D scenario, where the movements of users $u \in \mathbf{U}$ are restricted to a road network $G(\mathbf{V,E})$.
$\mathbf{V}$ is the set of vertices, where each vertice $v \in \mathbf{V}$ represents either a street intersection or an important landmark. $\mathbf{E}$ is the set of directed edges augmented by edge length and type. Edges are represented by a begin/end vertice pair and each edge represents the smallest unit of a road segment. $e \in \mathbf{E}$, each $e$ being a tuple specifying id, start-/end-vertices, length, and \rt $(e_{id},v_s,v_e,e_{length},e_{RT})$. \rt is a hierarchy of the size/type of road i.e. highway, paved, or dirt road (Sec.~\ref{subsec:roadtypes}).

The simplest form of trajectory is a collection of tuples $(time,longitude,latitude)$, ordered by the time attribute, but as we will work on a road network and in the spatio-temporal domain, such a basic notion of trajectories is not appropriate. 
We define $\mathbf{T}$ as the set of trajectories, where each trajectory consist of an id $(t_{id})$, and a sequence of tuples containing an edge and start-/end-time ($\tau_{s_i}/\tau_{e_i}$) of edge traversal. A trajectory is then  $(t_{id}, \langle (e_{_j},\tau_{s_j},\tau_{e_j}), \ldots,(e_{_k}, \tau_{s_k},\tau_{e_k}) \rangle)$, where $t_{id}  \in \mathbb{N}$, $\tau_{s_i},\tau_{e_i}  | \tau_{s_i} < \tau_{e_i} \wedge \tau_{s_{i-1}} < \tau_{s_{i}} \wedge \tau_{e_{i-1}} < \tau_{e_{i}}, \tau_{s_i} \in \mathbb{N}, \tau_{e_i} \in \mathbb{N}$, and $e \in \mathbf{E}$, the set of edges. We restrict trajectories to the road network, and users are assumed to traverse the entirety of an edge.

A user $u \in \mathbf{U}$ is defined by a tuple $(u_{id}, \{s\}, \{t\})$, $u_{id} \in \mathbb{N}$, $s \in \mathbf{S}$ the set of Privacy Profiles (Sec.~\ref{sec:usrprofile}), and $t \in \mathbf{T}$. 

Traditional ways of obtaining user privacy often include spacial obfuscation by including the user in a bounding box, or cloaking region \cite{trajecGeneral09, semantic08}. Conventional classification methods working on trajectories \cite{Li2007, outlier09, outlier08} are not suited to work on cloaking regions, and applying such anonymization techniques will break these algorithms or require them to be modified to still function correctly.

Our approach has identical input and output format and keep the format simple by just having a list of trajectory ids, each one associated with a list of edge ids and their start-/end traversal time. We will thus not require any modifications of the traditional approaches for them to work on the anonymized dataset. The input format chosen only require a map to have ids on road edges, a reasonable assumption of most map data.
After anonymization road edges may have been removed, and timestamps may have been modified, the output format is however still identical to the input format.

$\{\forall u \in \mathbf{U}\}$ specify at least one privacy profile. Each privacy profile specify the privacy settings for each section of road, such that for an edge  $e \in \mathbf{E}$ traversed by the user, the system will anonymize $e$ according to the privacy profile of $u$. Users need only specify one privacy profile, but may have more to cover different user contexts, e.g. work week, weekend, vacation.


We want to develop a solution for the described approach, accomplishing the following important goals:
\begin{itemize}
	\item {\bf Usability }
		To be useful, the privacy requirements has to be simple for the user to define.
	\item {\bf Practical }
		Besides when user specify his privacy profile, it should not require user interaction while running correctly.
	\item {\bf Flexible }
		Handle users specifying specific sensitive locations as well as specifying "everything but this" is sensitive.
\end{itemize}


% The proposed approach must be computationally efficient and the anonymized dataset must still contain enough data that the following two types of data analysis can be performed %with $i$, where $i$ is the percentage of extra errors introduced for each type of analysis after anonymization has been performed.
% \begin{itemize}
% 	\item Which sub-trajectories are popular when traveling from point A to point B.
% 	\item When and where does the road segments experience high load
% \end{itemize}

Given a set of user $u \in \mathbf{U}$ with a set of privacy profiles $s \in \mathbf{S}$ and a set of trajectories $t \in \mathbf{T}$ on a road network $G(\mathbf{V,E})$ the problem is then for each $u$ to apply $s$ on $\{t\}$.



