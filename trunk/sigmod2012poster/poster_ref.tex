\documentclass[a3shrink,landscape,final]{baposter}
%\documentclass[a4shrink,portrait,final]{baposter}
% Usa a4shrink for an a4 sized paper.

\tracingstats=2

\usepackage{times}
\usepackage{calc}
\usepackage{graphicx}
\usepackage{amsmath}
\usepackage{amssymb}
\usepackage{relsize}
\usepackage{multirow}
\usepackage{bm}

\usepackage{graphicx}
\usepackage{multicol}

\usepackage{pgfbaselayers}
\pgfdeclarelayer{background}
\pgfdeclarelayer{foreground}
\pgfsetlayers{background,main,foreground}

\usepackage{helvet}
%\usepackage{bookman}
\usepackage{palatino}

\newcommand{\captionfont}{\footnotesize}

\selectcolormodel{cmyk}

\graphicspath{{images/}}

%%%%%%%%%%%%%%%%%%%%%%%%%%%%%%%%%%%%%%%%%%%%%%%%%%%%%%%%%%%%%%%%%%%%%%%%%%%%%%%%
%%%% Some math symbols used in the text
%%%%%%%%%%%%%%%%%%%%%%%%%%%%%%%%%%%%%%%%%%%%%%%%%%%%%%%%%%%%%%%%%%%%%%%%%%%%%%%%
% Format 
\newcommand{\Matrix}[1]{\begin{bmatrix} #1 \end{bmatrix}}
\newcommand{\Vector}[1]{\Matrix{#1}}
\newcommand*{\SET}[1]  {\ensuremath{\mathcal{#1}}}
\newcommand*{\MAT}[1]  {\ensuremath{\mathbf{#1}}}
\newcommand*{\VEC}[1]  {\ensuremath{\bm{#1}}}
\newcommand*{\CONST}[1]{\ensuremath{\mathit{#1}}}
\newcommand*{\norm}[1]{\mathopen\| #1 \mathclose\|}% use instead of $\|x\|$
\newcommand*{\abs}[1]{\mathopen| #1 \mathclose|}% use instead of $\|x\|$
\newcommand*{\absLR}[1]{\left| #1 \right|}% use instead of $\|x\|$

\def\norm#1{\mathopen\| #1 \mathclose\|}% use instead of $\|x\|$
\newcommand{\normLR}[1]{\left\| #1 \right\|}% use instead of $\|x\|$

%%%%%%%%%%%%%%%%%%%%%%%%%%%%%%%%%%%%%%%%%%%%%%%%%%%%%%%%%%%%%%%%%%%%%%%%%%%%%%%%
% Multicol Settings
%%%%%%%%%%%%%%%%%%%%%%%%%%%%%%%%%%%%%%%%%%%%%%%%%%%%%%%%%%%%%%%%%%%%%%%%%%%%%%%%
\setlength{\columnsep}{0.7em}
\setlength{\columnseprule}{0mm}


%%%%%%%%%%%%%%%%%%%%%%%%%%%%%%%%%%%%%%%%%%%%%%%%%%%%%%%%%%%%%%%%%%%%%%%%%%%%%%%%
% Save space in lists. Use this after the opening of the list
%%%%%%%%%%%%%%%%%%%%%%%%%%%%%%%%%%%%%%%%%%%%%%%%%%%%%%%%%%%%%%%%%%%%%%%%%%%%%%%%
\newcommand{\compresslist}{%
\setlength{\itemsep}{1pt}%
\setlength{\parskip}{0pt}%
\setlength{\parsep}{0pt}%
}


%%%%%%%%%%%%%%%%%%%%%%%%%%%%%%%%%%%%%%%%%%%%%%%%%%%%%%%%%%%%%%%%%%%%%%%%%%%%%%
%%% Begin of Document
%%%%%%%%%%%%%%%%%%%%%%%%%%%%%%%%%%%%%%%%%%%%%%%%%%%%%%%%%%%%%%%%%%%%%%%%%%%%%%

\begin{document}

%%%%%%%%%%%%%%%%%%%%%%%%%%%%%%%%%%%%%%%%%%%%%%%%%%%%%%%%%%%%%%%%%%%%%%%%%%%%%%
%%% Here starts the poster
%%%---------------------------------------------------------------------------
%%% Format it to your taste with the options
%%%%%%%%%%%%%%%%%%%%%%%%%%%%%%%%%%%%%%%%%%%%%%%%%%%%%%%%%%%%%%%%%%%%%%%%%%%%%%
% Define some colors
\definecolor{silver}{cmyk}{0,0,0,0.3}
\definecolor{yellow}{cmyk}{0,0,0.9,0.0}
\definecolor{reddishyellow}{cmyk}{0,0.22,1.0,0.0}
\definecolor{black}{cmyk}{0,0,0.0,1.0}
\definecolor{darkYellow}{cmyk}{0,0,1.0,0.5}
\definecolor{darkSilver}{cmyk}{0,0,0,0.1}

\definecolor{lightyellow}{cmyk}{0,0,0.3,0.0}
\definecolor{lighteryellow}{cmyk}{0,0,0.1,0.0}
\definecolor{lightestyellow}{cmyk}{0,0,0.05,0.0}

\definecolor{white}{cmyk}{0,0,0,0}
\definecolor{blue}{cmyk}{0.74,0.27,0,0}
\definecolor{lighterblue}{cmyk}{0.85,0.0,0.0,0.0}
\definecolor{lightestblue}{cmyk}{0.09,0,0,0}
\definecolor{borderblue}{cmyk}{0.74,0.27,0,0.4}
\definecolor{headerblue}{cmyk}{0.74,0.27,0,0.5}
\definecolor{bgblue}{cmyk}{0.9,0.5,0.1,0}
\definecolor{bgblue2}{cmyk}{0,0,0,0}
\definecolor{boxColor1}{cmyk}{0.1,0.05,0,0}


%%
\typeout{Poster Starts}
\background{
  \begin{tikzpicture}[remember picture,overlay]%
    \draw (current page.north west)+(-2em,2em) node[anchor=north west] {\includegraphics[height=1.1\textheight]{images/silhouettes_background}};
  \end{tikzpicture}%
}

\newlength{\leftimgwidth}
\begin{poster}%
  % Poster Options
  {
  % Show grid to help with alignment
  grid=no,
  % Column spacing
  colspacing=1em,
  % Color style
%  bgColorOne=lightestyellow,
%  bgColorTwo=lightestyellow,
%  borderColor=reddishyellow,
%  headerColorOne=yellow,
%  headerColorTwo=reddishyellow,
%  headerFontColor=black,
%  boxColorOne=lightyellow,
%  boxColorTwo=lighteryellow,
  bgColorOne=bgblue2,
  bgColorTwo=bgblue,
  borderColor=borderblue,
  headerColorOne=headerblue,
  headerColorTwo=lightestblue,
  headerFontColor=white,
  boxColorOne=white,
  boxColorTwo=boxColor1,
  % Format of textbox
  textborder=roundedleft,
  % Format of text header
  eyecatcher=no,
  headerborder=open,
  headerheight=0.08\textheight,
  headershape=roundedright,
  headershade=shade-lr,
  headerfont=\Large\textsf, %Sans Serif
  boxshade=shade-lr,
  background=shade-tb,
%  background=plain,
  linewidth=2pt
  }
  % Eye Catcher
  {\includegraphics[width=8.5em]{images/daisy-logo}} % eye catcher for this poster. (eyecatcher=no above). If an eye catcher is present, the title is centered between eye-catcher and logo.
  % Title
  {\sf %Sans Serif
  %\bf% Serif
  A Location Privacy Aware Friend Locator}
  % Authors
  {\sf %Sans Serif
  % Serif
  \vspace{0.1em} Laurynas \v{S}ik\v{s}nys, Jeppe R. Thomsen, Simonas \v{S}altenis, Man Lung Yiu, Ove Andersen\\
  {\vspace{0.2em}\large Department of Computer Science, Aalborg University}
  }
  %{\sf
  %\normalsize Department of Computer Science, Aalborg University}
  % University logo
  {% The makebox allows the title to flow into the logo, this is a hack because of the L shaped logo.
    \makebox[8em][r]{%
      \begin{minipage}{16em}
        \hfill
        \includegraphics[height=5.5em]{images/polyu_logo_l}
      \end{minipage}
    }
  }

  \tikzstyle{light shaded}=[top color=baposterBGtwo!30!white,bottom color=baposterBGone!30!white,shading=axis,shading angle=90]

  % Width of left inset image
     \setlength{\leftimgwidth}{0.78em+8.0em}

%%%%%%%%%%%%%%%%%%%%%%%%%%%%%%%%%%%%%%%%%%%%%%%%%%%%%%%%%%%%%%%%%%%%%%%%%%%%%%
%%% Now define the boxes that make up the poster
%%%---------------------------------------------------------------------------
%%% Each box has a name and can be placed absolutely or relatively.
%%% The only inconvenience is that you can only specify a relative position 
%%% towards an already declared box. So if you have a box attached to the 
%%% bottom, one to the top and a third one which should be in between, you 
%%% have to specify the top and bottom boxes before you specify the middle 
%%% box.
%%%%%%%%%%%%%%%%%%%%%%%%%%%%%%%%%%%%%%%%%%%%%%%%%%%%%%%%%%%%%%%%%%%%%%%%%%%%%%
    %
    % A coloured circle useful as a bullet with an adjustably strong filling
    \newcommand{\colouredcircle}[1]{%
      \tikz{\useasboundingbox (-0.2em,-0.32em) rectangle(0.2em,0.32em);
\draw[draw=black,fill=borderblue!80!black!#1!white,line width=0.03em] (0,0)
circle(0.18em);}}

    \newenvironment{indentpar}[1]%
    { \vspace{-1em}
      \begin{list}{}%
	    {\setlength{\leftmargin}{#1}}%
	    \item[]%
    }
    {\end{list} \vspace{-1em}}

%%%%%%%%%%%%%%%%%%%%%%%%%%%%%%%%%%%%%%%%%%%%%%%%%%%%%%%%%%%%%%%%%%%%%%%%%%%%%%
  \headerbox{Overview}{name=contribution,span=1.5,column=0,row=0}{
%%%%%%%%%%%%%%%%%%%%%%%%%%%%%%%%%%%%%%%%%%%%%%%%%%%%%%%%%%%%%%%%%%%%%%%%%%%%%%
     \colouredcircle{100} \textbf{Applications:} the proposed centralized system notifies a user if he is geographically close to any of his friends.

     \colouredcircle{100} \textbf{Privacy:} strong user location privacy is
guaranteed via the encryption of user locations mapped to a spatial grid. The
server can not deduce actual locations of any users. It can only detect
proximity or separation between user pairs.

     \colouredcircle{100} \textbf{Communication cost:} communication is
optimized such that a client is not required to send any data if the user is far
away from his closest friend.

     \colouredcircle{100} \textbf{Experimental results:} the proposed solution
is efficient and scalable to a large number of users.
	 
  \vspace{0.5em}
 }

%%%%%%%%%%%%%%%%%%%%%%%%%%%%%%%%%%%%%%%%%%%%%%%%%%%%%%%%%%%%%%%%%%%%%%%%%%%%%%
  \headerbox{Functionality}{name=function,column=0,span=1.5,below=contribution}{
%%%%%%%%%%%%%%%%%%%%%%%%%%%%%%%%%%%%%%%%%%%%%%%%%%%%%%%%%%%%%%%%%%%%%%%%%%%%%%
		 The central server sends notification to users $\MAT{u_1}$ and $\MAT{u_2}$ if the 
		 proximity is detected according to the following conditions:		

		 \begin{indentpar}{0.5cm}
		  \colouredcircle{100}  if $dist(\MAT{u_1}, \MAT{u_2})$ $\leq$ 
		      $\epsilon$ then $\MAT{u_1}$ and $\MAT{u_2}$ are in proximity;
		  		  		 
		  \colouredcircle{100}  if $dist(\MAT{u_1}, \MAT{u_2})$ $\geq$
		  $\epsilon + \lambda$ then the users $\MAT{u_1}$ and $\MAT{u_2}$ are not in
		  proximity;
				
		  \colouredcircle{100} if $\epsilon$ $<$ $dist(\MAT{u_1}, \MAT{u_2})$ $<$ $\epsilon$ + $\lambda$ then the server freely classifies 
		  users $\MAT{u_1}$ and $\MAT{u_2}$ as being either in or not in proximity.
		\end{indentpar}

 		Here, $dist(\MAT{u_1}, \MAT{u_2})$ is the Euclidean distance between the users $\MAT{u_1}$ and $\MAT{u_2}$. The 
 		$\epsilon$ is the proximity detection distance, which is selected by users $\MAT{u_1}$ and $\MAT{u_2}$ from 
 		a set of discrete values. The parameter	$\lambda = \epsilon(2\sqrt{2}-1)$ is the precision parameter of the service.		
  \vspace{0.5em}
  }

  \headerbox{Grid-based Encryption}{name=questions,column=0,span=1.5,below=function,above=bottom}{
		\begin{center}
		   \includegraphics[width=0.95\linewidth]{images/LRS_drawings2}   
		\end{center}
   \vspace{0px} 

User locations are mapped into four cells of a uniform spatial
grid. These mappings are encrypted with an encryption function $\Psi$
and then sent to the server. The function $\Psi$ is shared among small
group of users that is a subset of all users in the system. If there
is a match between the encrypted values of two users, the server
either detects a proximity or asks them to use a finer grid. The
proximity is detected only if the grid is fine enough to match
$\epsilon$ setting of the two users. This process is called the
\textit{Incremental Proximity Detection}.  }

%%%%%%%%%%%%%%%%%%%%%%%%%%%%%%%%%%%%%%%%%%%%%%%%%%%%%%%%%%%%%%%%%%%%%%%%%%%%%%
  \headerbox{Incremental Proximity Detection}{name=incremental proximity detection,column=1.5,span=2.5,row=0}{
%%%%%%%%%%%%%%%%%%%%%%%%%%%%%%%%%%%%%%%%%%%%%%%%%%%%%%%%%%%%%%%%%%%%%%%%%%%%%%
  
  \includegraphics[width=\linewidth]{images/dgDemo}
 \vspace{0.5em}

	This example illustrates the behavior of the incremental
        proximity detection algorithm. The figure shows the
        geographical locations of users \MAT {u_1} and \MAT {u_2} and
        their mappings into the list of grids at 4 snapshots in
        time. Users \MAT {u_1} and \MAT {u_2} have matching cells at
        level 0~(a), so both \MAT{u_1} and \MAT{u_2} are asked to
        switch to level 1, where they no longer have a match.  At the
        second snapshot (b) both users have moved, but only \MAT{u_1}
        has changed grid cells, so only \MAT{u_1} resends his
        encrypted location for level 0. The new encrypted cells match
        at the server and so \MAT{u_1} is asked to send his location
        for level 1 (c).  As both \MAT {u_1} and \MAT {u_2} still
        have a match at level 1, they both switch to level 2 (d), and,
        as this level corresponds to their desired $\epsilon$ setting,
        they both receive a proximity notification.
  \vspace{0.5em}
  }

%%%%%%%%%%%%%%%%%%%%%%%%%%%%%%%%%%%%%%%%%%%%%%%%%%%%%%%%%%%%%%%%%%%%%%%%%%%%%%
  \headerbox{Results}{name=results,column=1.5,span=2.5,above=bottom,below=incremental proximity detection}{
%%%%%%%%%%%%%%%%%%%%%%%%%%%%%%%%%%%%%%%%%%%%%%%%%%%%%%%%%%%%%%%%%%%%%%%%%%%%%%

      	Our approach was evaluated on a workload generated over
      	the German city of Oldenburg with 50000 users partitioned into 200 disjoint groups.
      	\vspace{0.6em}
      	
      \begin{tabular}{cc}	
      	\begin{tabular}{c}
						\includegraphics[width=0.23\linewidth]{images/basic} \vspace{0.2em}\\ 
						(a)
				\end{tabular}
	      \begin{tabular}{cc}
						\includegraphics[width=0.33\linewidth]{images/epsilonEffect2} &
						\includegraphics[width=0.33\linewidth]{images/allMsgOneUser} \\
						(b) & (c)
				\end{tabular} &
      \end{tabular}
      \vspace{1em}
        
        For comparison, an alternative approach, called {\em basic}, is used.
        The {\em basic} approach detects proximity based on the filter-and-refine paradigm. The server
        measures the maximum and minimum distances between user location cells of size $d$ (See (a)),
        and either detects proximity/separation or asks the users to invoke a peer-to-peer refinement step.
        
        
        Figure (a) shows how the communication cost is affected by the
        changing $\epsilon$. The cost of our approach is robust to
        different values of $\epsilon$ and, unlike {\em basic}, its
        cost rises slowly when $\epsilon$ increases.
      	
        Figure (c) benchmarks our approach against {\em basic} with
        different cell sizes, representing different privacy
        levels. For our approach, in contrast to {\em basic},
        increasing the number of users in each disjoint group has
        almost no impact on the amount of messages a user has to send
        to the server.

        %larger than 150 users.
        
       
       
       %The baseline approach detects proximity based on a max and minimum distance, 
       %with a cell size $d$. The proximity is detected if the distance between two
       %friends is $\leq maxDist$, it does not detect proximity if the distance is $< minDist$       

        

  \vspace{0.5em}
  }
%%%%%%%%%%%%%%%%%%%%%%%%%%%%%%%%%%%%%%%%%%%%%%%%%%%%%%%%%%%%%%%%%%%%%%%%%%%%%%%
%  \headerbox{Robustness}{name=robustness,column=3,row=0,above=results,span=1}{
%%%%%%%%%%%%%%%%%%%%%%%%%%%%%%%%%%%%%%%%%%%%%%%%%%%%%%%%%%%%%%%%%%%%%%%%%%%%%%%
%  \begin{tikzpicture}[x=0.3333\linewidth,y=-0.42\linewidth]
%    \path [use as bounding box] (-0.5,-0.5) rectangle(2.5,1.7);
%    \path
%    (0,0) node{\includegraphics[width=0.42\linewidth]{D1160}}
%    (1,0) node{\includegraphics[width=0.42\linewidth]{D1425}}
%    (2,0) node{\includegraphics[width=0.42\linewidth]{D1205}}
%
%    (0,1) node{\includegraphics[width=0.28\linewidth]{D1160_fit_expression}}
%    (1,1) node{\includegraphics[width=0.28\linewidth]{D1425_fit_expression}}
%    (2,1) node{\includegraphics[width=0.28\linewidth]{D1205_fit_expression}}
%
%    (1,0.5) node {\smaller a) Target}
%    (1,1.6) node {\smaller b) Robust Reconstruction};
%  \end{tikzpicture}
%  \vspace{0.5em}
%
%  The reconstruction (b) is robust against scans (a) with artifacts, noise, and
%  holes.
%  
%  This is achieved by a robust iteratively reweighted ICP algorithm and outlier
%  rejection based on angle comparisions between corresponding points.
%  }

%%%%%%%%%%%%%%%%%%%%%%%%%%%%%%%%%%%%%%%%%%%%%%%%%%%%%%%%%%%%%%%%%%%%%%%%%%%%%%%
%  \headerbox{References}{name=references,column=0,span=1.5,above=bottom}{
%%%%%%%%%%%%%%%%%%%%%%%%%%%%%%%%%%%%%%%%%%%%%%%%%%%%%%%%%%%%%%%%%%%%%%%%%%%%%%%
%    \smaller
%    \vspace{-0.4em}
%    \bibliographystyle{ieee}
%    \renewcommand{\section}[2]{\vskip 0.05em}
%      \begin{thebibliography}{1}\itemsep=-0.01em
%      \setlength{\baselineskip}{0.4em}
%      \bibitem{amberg07:nonrigid}
%        B.~Amberg, S.~Romdhani, T. Vetter.
%        \newblock {O}ptimal {S}tep {N}onrigid {ICP} {A}lgorithms for {S}urface {R}egistration
%        \newblock In {\em Computer Vision and Pattern Recognition 2007}
%      \bibitem{amberg08:recognition}
%        B.~Amberg, R.~Knothe, T. Vetter.
%        \newblock Expression Invariant Face Recognition with a 3D Morphable Model
%        \newblock In {\em Automated Face and Gesture Recognition 2008}
%      \end{thebibliography}
%  }
%%%%%%%%%%%%%%%%%%%%%%%%%%%%%%%%%%%%%%%%%%%%%%%%%%%%%%%%%%%%%%%%%%%%%%%%%%%%%%

\end{poster}%
\end{document}
