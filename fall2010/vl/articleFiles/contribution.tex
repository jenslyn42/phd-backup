\section{Our privacy-aware proximity based service} \label{sec:contribution}

In this section we introduce base concepts employed in our proximity detection
service, followed by client and server algorithms. 

\subsection{Proximity detection idea}

This section describes how the \ls can locate users in their friend's vicinity
without disclosing their locations and vicinities. 

Similarly to \textit{Incremental Proximity Detection Approach} \cite {ffinder},
where all users in $\mathbf{M}$ share a list of grids, here we let all users in
$\mathbf{M}$ share a \textit{list of granularities}. The list of granularities,
denoted by $\Gamma$, contain finite or infinite number of granularities
$\Gamma(l)| l=0, 1, 2,...$  A single granularity
%\footnote{ Note, that a grid is a special case of granularity} 
specifies a subdivision of the spatial domain
into a number of non-overlapping regions, called granules\cite{pbsPaper}. The
granularity's index $l \geq 0$ in the $\Gamma$ is termed \textit{the level of
granularity}. Every granularity $\Gamma(l)$ at levels $l=0, 1, 2,...$ satisfies
following three properties:
\begin{itemize}
 \item Every granule $g \in \Gamma(l)$ is identifiable by an index $id(g)
\in \mathbb{N}$.
 \item Every granule $g \in \Gamma(l)$ has a bounded, not higher than $L(l)$,
size, i.e. $\forall g \in \Gamma(l), MaxDist(g) \leq L(l)$, where $MaxDist(g)$
is maximal Euclidean distance between any two points of region $g$.
 \item Granule sizes bound $L(l)$, in level $l>0$, is always lower than in level
$l-1$, i.e. $L(l) < L(l-1)$.
 \item Every granule in level $l>0$ is fully contained by some granule at 
level $l-1$.
\end{itemize}

Figure \ref{fig:listOfGrids}a visualizes a valid granularity
list, where uniform grids are used as granularities and grid cells are used as
granules at levels $0$ to $2$. Note, that it only shows subsets of all
available cells for every grid. The ``top-view'' projection of this list is
provided in Fig. \ref{fig:listOfGrids}b. Solid, dashed and dotted lines depict
boundaries of cells at levels $0$, $1$, $2$ respectively. Every cell $c$ in levels
$l=0,1,2$ has defined $id(c)$ and its size $MaxDist(g)$ is lower-equal to $L(l)$. 
For example, maximal distances between any two points of some 
cells $c0$,$c1$, $c2$ at levels $0$,$1$,$2$ respectively are $70.71$, $35.36$, 
$17.68$ and they correspond to levels' $L$ values in the uniform grids case. 
Moreover, cells of lower level grids fully contain cells of higher level grids.

\begin{figure}
       \center
       \begin{tabular}{c c c}			   
	  \includegraphics[width=0.16\textwidth]{vl/images/dgLMGlist2.pdf} &	
	  \includegraphics[width=0.14\textwidth]{vl/images/dgLMGlist3.pdf} & 
	  \includegraphics[width=0.12\textwidth]{vl/images/proxDetVL.pdf} \\
          (a) & (b) & (c)
       \end{tabular}
       \caption{The valid list of granularities and the
behavior of the granularity-based classifier}
            \label{fig:listOfGrids}
\end{figure}

Let us assume, that a list of granularities $\Gamma$ is globally defined in the
system and thus fixed on all clients. 

Similarity to FriendLocator \cite {ffinder}, we let all users in $\mathbf{M}$
also share an encryption function $\Psi$. $\Psi: \mathbb{N} \mapsto \mathbb{N}$
is one-to-one function that is used to map index $id(g)$ of some granule $g$ to
corresponding encrypted representation. In practice $\Psi$ can be implemented as
a keyed secure hash function (e.g. SHA-2) such that it is computationally
infeasible for the attacker to break. A key of the hash function can be
distributed among clients of $\mathbf{M}$ in peer-to-peer fashion or with help
of some trusted third-party server.

When $\Psi$ is known by clients, but not by the \ls, each client utilizing
$\Psi$ can encrypt indices of granules that enclose their location and vicinity
at some granularity. Encrypted representation of these indices can be compared
on the \ls without need to disclose indices of granules and consequently the
vicinity or location of any user. In particular, assume that two friends
$u_1$, $u_2$ use granularity of some level $l$ (Fig. \ref{fig:listOfGrids}c) and
the user $u_2$ finds his location containing granule $g_l$, applies $\Psi$ on
its index $id(g_l)$ and sends encrypted representation, $e123$, to the \ls.
Similarly user $u_1$ finds all his vicinity intersecting granules
$\mathbf{g_v}$, applies $\Psi$ on their indices and send their encrypted
representations, \{$e342$, $e433$, $e034$, $e211$, $e987$, $e123$\}, to the
\ls. If encrypted index of $u_2$'s location granule, $e123$, can be found in
encrypted indices set of user $u_1$ vicinity then we can conclude that user
$u_2$ is in $u_1$'s vicinity. Let us call such location-vicinity intersection 
detection approach by \textit{granularity-based classifier} and, based on the 
knowledge about the list of granularities, Lemma \ref{lem1} can be derived.

\begin{lemma}
\label{lem1}
 The granularity-based classifier can be used to classify the user $u_2$ as being
in $u_1$'s proximity or not with the precision parameter setting
$\lambda=L(l)$, defined in Section \ref{sec:problemdef}.
\end{lemma}
\begin{proof}
 According the Section \ref{sec:problemdef}, in order for user $u_2$ to be in
$u_1$'s proximity or not in proximity, respective conditions
$distLV(loc(u_2),vic(u_1)) \leq \lambda$, and $loc(v) \notin vic(u)$ must hold.
If the encrypted index of $u_2$'s location granule, e.g. $e123$, can be found in
encrypted indices set of user $u_1$ vicinity, due to $\Psi$ is one-to-one
mapping we can conclude that $u_2$ location containing granule $g_l$ is in
$u_1$'s vicinity intersecting granules set $\mathbf{g_v}$. Then we know
that granule $g_l$ both encloses $u_2$ location $loc(u_2)$ and intersects with
$u_1$ vicinity $vic(u_1)$. Utilizing properties of granularity at level $l$,
we know that maximal Euclidean distance between any two points within granule
$g_l$ is lower-equal than $L(l)$, i.e. $MaxDist(g_l) \leq L(l)$. Thus the shortest
Euclidean distance between location $loc(u_2)$ and the vicinity region
$vic(u_1)$ cannot be higher than $L(l)$, i.e. $distLV(loc(u_2),vic(u_1)) \leq
\lambda$. Similarly we can prove that if $g_l$ cannot be found in
$\mathbf{g_v}$ then $loc(v) \notin vic(u)$.
\end{proof}

According to Lemma \ref{lem1}, the $\lambda$ value depends on granule sizes
bound function $L$ and the granularity level $l$. We can observe that higher
levels provide higher proximity detection precision such that $\lim_{l
\to \infty} \lambda = \lim_{l \to \infty} L(l) = 0$. However as we go to
higher levels the number of
vicinity intersecting granules increases causing higher client
communication. Thus we let for every user $u \in \mathbf{M}$ to select a
constant $L_{max}(u)$ that has following meanings:
\begin{itemize}
 \item User $u$ will never use granularities of higher than
$L_{max}(u)$ levels thus limiting his worts case communication.
 \item User $u$ lets other users to detect him in a proximity with no higher
than $\lambda=L(L_{max}(u))$ precision. Thus $L_{max}(u)$ can be used to 
specify $u$'s minimum location privacy requirements.
 \item User $u$ will be able to detect friends being in his proximity with no
higher than $\lambda=L(L_{max}(u))$ precision.
 \item Every encrypted coordinate of user that is sent to \ls will have no
higher than $L(L_{max}(u))$ resolution, i.e., if the attacker brakes
the encrypted coordinate, then deciphered value will correspond
to cloaking region with maximum distance between two points no lower than
$L(L_{max}(u))$.
\end{itemize}

Users can freely select such $L_{max}$ at runtime and upload it to the \ls.
Note that this does not violate user location privacy as it does not reveal any
spatial information.

The introduced granularity-based classifier is integrated into our proximity
detection service which is defined using client and server algorithms in the
following section.


\subsection{Client and Server algorithms}

We define our proximity based service by providing handler algorithms for
different type of software events on a \md and the \ls, in particular:


- \textbf{onMsgReceived($msg$, $arg$)}
	A handler is executed on a \md or the \ls each time one receives a message
of type $msg$ with arguments $arg$ from other party. A summarizing list of
employed messages with their arguments is presented in Table \ref{tblMessages}.


- \textbf{onLocChange()}
 A handler executed on a \md, each time its geographical location changes.

\vspace{0.25cm}


\begin{table}
	\centering
	\begin{tabular}{| p{0.95cm} | p{0.7cm} | p{0.6cm}| p{4cm} |}
		 \hline
	Message Type & Args & Sender & Description \\ \hline
			  
	$M_{el}$ & $u$, $l$, $g^*_l$, $\mathbf{g^*_v}$ & \md & \md with id equal
to $u$ sends this type of message to the \ls in order to report his encrypted
location $g^*_l$ and the vicinity $\mathbf{g^*_v}$ for level $l$. \\ \hline

	$M_{prox}$ & $v, l$ & \ls & The \ls sends this type of message to a \md
to inform that his friend $v$ is within his vicinity at granularity level $l$.
\\ \hline

	$M_{LevInc}$ & $l$ & \ls & The \ls sends this type of message to some
\md to make it increase its level up to level $l$. \\ \hline
			  
	\end{tabular}
	\caption{Client and server messages types}\label{tblMessages}
\end{table}


\begin{algorithm}[!Hbt]
    \dontprintsemicolon
    \SetVline
\footnotesize
    \KwData{
  
\footnotesize 
    
  $u \in \mathbf{M}$ - current user ID.

	$loc(u)$ - user's current location.
	
  $vic(u)$ - user's vicinity region.

  $GS$ - Stack of 2-tuples $\langle g_l$, $\mathbf{g_v} \rangle$, where positions on the stack corresponds to levels.
    	$g_l$ is $loc(u)$ granule index at level $l$. 
    	$\mathbf{g_v}$ is a indices set of $vic(u)$ intersecting granules at level $l$.
			
	$L_{max}(u)$ - user specified highest granularity level.
    }

\footnotesize 
    \funcc{mapLocToGranularity}{\text{Level number } level}
    {
	$g_l \leftarrow \{id(g) | g \in \Gamma(level): loc(u) \in g\}$ ;

	$\mathbf{g_v} \leftarrow \{id(g) | \forall g \in \Gamma(level): g
\cap vic(u) \neq \emptyset\}$;

	\textbf{return} ($g_l$, $\mathbf{g_v}$);
    }

\footnotesize
    \funcc{pushAndSend}{}
    {
	($g_l$, $\mathbf{g_v}$) $\leftarrow$ $mapLocToGranularity(|GS|)$;

        Push $\langle g_l$, $\mathbf{g_v} \rangle$ to stack $GS$;

        Send to \ls $M_{el}(u, |CS|-1, \Psi(g_l), \{\Psi(g)| \forall g \in
\mathbf{g_v}\})$;
    }

\footnotesize 

    \funcc{onLocChange}{}
    {
        $wasPopped \leftarrow false$

        \While {$|GS| > 0$ \text{ and } $top(GS) \neq
mapLocToGranularity(|CS|-1)$}
        {
            Pop from stack $GS$;

            $wasPopped \leftarrow true$;
        }

        \If {$wasPopped$ \text{ or } $|GS| = 0$}
        {
            \textbf{pushAndSend}()
        }
    }

    \funcc{onMsgReceived}{M_{LevInc}, \text{Level } l}
    {
        \While{$|GS| \leq l$ \text{ and } $|GS| \leq L_{max}(u)$}
        {
            \textbf{pushAndSend}()
        }
    }

    \funcc{onMsgReceived}{M_{prox}, \text{Friend }  v, \text{Level} l}
    {
        Output "Friend ",$v$," with precision ", $L(l)$, " is inside
our vicinity!";
    }

\caption{The \md's event handlers in our proximity based service.} 
\label{algCl}
\end{algorithm}



Algorithms \ref{algCl} and \ref{algSrv} specify the behavior of the \md and the
\ls. They contains local data definitions, functions and software
event handlers. 

A \md $u$ remembers his last positioning unit reported geographical location
$loc(u)$, and for granularity levels $l=0..(|GS|-1)$ (see Alg. \ref{algCl}) locally stores his
location and vicinity mappings, i.e., indices of location and vicinity granules,
in the stack $GS$. Once \md changes its location, the \textbf{onLocChange}
handler is triggered. Then if user's location change invalidates current
location or vicinity granules at levels $l=(|GS|-1) .. 0$, the \md removes
respective elements from $GS$, thus reducing his employed current level
$|GS|-1$. Note, that this corresponds to zero or more $u$'s switches from finer
to coarser granularities. At least one current level reduction is always followed by
$\textbf{pushAndSend}$ call, which computes new location and vicinity mappings
for $|GS|$ (current + one level) and sends them to the \ls.
 
Client granularity level shifts can be visualized by Fig. \ref{fig:VLdemo}a and
\ref{fig:VLdemo}b, where user's $u_1$ vicinity and user's $u_2$ current location
mapping into list of granularities (grids) are visualized at consecutive time
snapshots. Note, that due to simplicity $u_1$'s current location and $u_2$'s
vicinity mappings are not shown. User $u_1$ changes his location and shifts
from level 1 to level 0 (Fig. \ref{fig:VLdemo}a and \ref{fig:VLdemo}b) because
his location change invalidates his vicinity mappings at levels 1 and 0.
In contrary, $u_2$ location change causes no location mappings changes in levels
1 and 0, thus he stays in level 1.

For every user $u$ the \ls locally stores $GL(u)$, which is an encrypted
alternative for $GS$. It contains encrypted representations of $u$'s location
and vicinity granule indices for levels $0 .. GL(u) - 1$. The $u$'s stack $GS$
is synchronized with $GL(u)$ with help of $M_{el}$ message. When the \ls
received this type of message, then the handler $\mathbf{onMsgReceived}$  is
executed. It first updates the $GL(u)$ and later checks if any of $u$'s friends
entered its vicinity or if user $u$ entered his friends vicinities. This is
checked by searching if encrypted location granule $g^*_l$ can be found in the
set of encrypted vicinity granules $\mathbf{g^*_v}$ for some friend $f$ at some
level $l_m$. The level $l_m$ is the highest level, available in $GL(u)$
and $GL(v)$ that does not exceed $L_{max}(u)$ and $L_{max}(v)$. If $g^*_l$ is
found in the $\mathbf{g^*_v}$ but the $l_m$ is lower than $L_{max}(u)$ and
$L_{max}(v)$, it means than the proximity detection precision can be still
be increased as users specified $L_{max}$ values are not yet reached, thus the
\ls sends $M_{LevInc}$ to one or both users, asking them to increase they
current levels. Otherwise, if $g^*_l$ is found in the $\mathbf{g^*_v}$ and $l_m$
is equal to $L_{max}(u)$ or $L_{max}(v)$ then the \ls sends $M_{prox}$ message,
informing a user about a presence of friend in his vicinity.


Let us assume that $L_{max}(u_1) = L_{max}(u_2) = 2$ in Fig. \ref{fig:VLdemo}
example. The server finds that $g^*_l$ of user $u_2$ lays in $\mathbf{g^*_v}$
of user $u_1$ at level 0 in Fig. \ref{fig:VLdemo}a, thus due to $0 = l_m$ is
lower than $L_{max}(u_1)$ or $L_{max}(u_2)$ it sent $M_{LevInc}$ messages to
both users asking them to increase their current levels. When they both deliver
level 1 encrypted coordinates, that are higher than level 0 precision, the \ls
no longer founds $g^*_l$ in $\mathbf{g^*_v}$ and then nothing happens until one
of them starts moving. When $u_1$ sends his location data for level 0 in Fig.
\ref{fig:VLdemo}b, the $l_m$ is set to $0$ and due to it is lower than
$L_{max}(u_1)$ or $L_{max}(u_2)$ and user $u_2$ is at level 1 already, only
the user $u_1$ is asked to switch to level 1. Similarly, when the \ls finds
$g^*_l$ in $\mathbf{g^*_v}$ at level 1 in Fig. \ref{fig:VLdemo}c it asks both
users increase their levels as $1 = l_m$ is still lower than $L_{max}(u_1)$ or
$L_{max}(u_2)$. When two users deliver their encrypted location data to the \ls
for level $2$ in Fig. \ref{fig:VLdemo}d, the $l_m = 2$ is equal to
$L_{max}(u_1)$ and $L_{max}(u_u)$, then the user $u_1$ is informed about $u_2$
proximity with message $M_{prox}$.

\begin{figure}
       \center
        \includegraphics[width=0.4\textwidth]{vl/images/dgDemoVL.pdf}
        \caption{Example of level changes and proximity detection within \vl} 
       \label{fig:VLdemo}      
\end{figure}

Note that similarly to the \ff \cite{ffinder}, the presented algorithms
implement a kind of adaptive region-based up-date policy. The clients updates
their encrypted location data on the \ls only when location change triggers the
change of location or vicinity granularity mappings, i.e. user location and
vicinity enclosing granules, at their current levels. And if some user is far
away from his friends, then he or she stays at a low-level granularity with large
cells, which results in few encrypted location data updates as the user moves.
Only when the user approaches one of the friends, is he asked to switch to
higher levels with smaller granules. Thus, at a given time point, the users
current communication cost is not affected by the total number of his or her
friends, but by the distance of the closest friend. 

Next we review several optimizations possibility that can help reduce client
communication and server computation costs, followed by a technique to
minimize privacy leakage if encryption function $\Psi$ is intercepted by an
adversary.

\begin{algorithm}[!Hbt]		
    \dontprintsemicolon
    \SetVline
\footnotesize
    \KwData
    {    
\footnotesize

	$\mathbf{M}$ - a set of users;        

	$\mathbf{F}$ - a friendship relation that represents a social network. 
			
	$L_{max}(u) \forall u \in \mathbf{M}$ - highest allowed granularity level specified by user $u$.
                
  $GL(u) \forall u \in \mathbf{M}$ - a stack of 2-tuples $\langle g^*_l,
\mathbf{g^*_v} \rangle$, where every 2-tuple corresponds to level $l$.
For level $l$ the $g^*_l$ is an encrypted index of granule containing $u$'s location. 
The $\mathbf{g^*_v}$ is a indices set of encrypted granules, that intersect 
$u$'s vicinity.
      
        $\mathbf{P(u)} \subseteq \mathbf{M}$ - a set of $u \in \mathbf{M}$'s
currently proximate friends.
	 }        

\footnotesize 

   \funcc{onMsgReceived}{M_{el}, \text{User } u, \text{Level }l, g^*_l, \mathbf{g^*_v}}
    {
        \While {$|GL(u)| > 0$ \text{ and } $|GL(u)|$ > $l$}
        {
            Pop from stack $GL(u)$;
        }

        Push $\langle g^*_l, \mathbf{g^*_v} \rangle$ to stack $GL(u)$;

        \ForEach {$v \in \mathbf{M}$ \text{ such that } $v \neq u$ \text{ and }
$|GL(v)| > 0$ \text{ and } $(u,v) \in \mathbf{F}$}
        {
            $l_m \leftarrow$ $min$($|GL(u)|-1$, $|GL(v)|-1$, $L_{max}(u)$,
$L_{max}(v)$);

	    $vInU$ $\leftarrow$ $get(GL(v), l_m).g^*_l \in get(GL(u),
l_m).\mathbf{g^*_v}$;

	    $uInV$ $\leftarrow$ $get(GL(u), l_m).g^*_l \in get(GL(v),
l_m).\mathbf{g^*_v}$;

            \If {$vInU=true$ \text{ or } $uInV=true$ }
            {
                \If {$l_m$ = $L_{max}(u)$ \text{ or } $l_m$ = $L_{max}(v)$ }
                {
                    \If{$vInU$ \text{ and } $v \notin \mathbf{P(u)}$}
                    {
                    insert $v$ into $\mathbf{P(u)}$;
                    
                    send $M_{prox}(v, l_m)$ to \md $u$;
                    
                    }

                    \If{$uInV$ \text{ and } $u \notin \mathbf{P(v)}$}
                    {
                    insert $u$ into $\mathbf{P(v)}$;
                    
                    send $M_{prox}(u, l_m)$ to \md $v$;
                    
                    }
                }
                \Else
                {
                    \If {$l_m = |GL(u)|-1$}
                    {
                        send $M_{LevInc}$($l_m + 1$) to \md $u$
                    }
                    \If {$l_m = |GL(v)|-1$}
                    {
                        send $M_{LevInc}$($l_m + 1$) to \md $v$
                    }
                }
            }
	    \If {$vInU = false$}
	    {
		remove $v$ from $\mathbf{P(u)}$;
	    }
	    \If {$uInV = false$}
	    {
		remove $u$ from $\mathbf{P(v)}$;
	    }
        }
    }
    
\caption{Event handler on the \ls.} \label{algSrv}
\end{algorithm}



