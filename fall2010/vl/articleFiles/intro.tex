\section{Introduction} \label{sec:intro}
%
Mobile devices with geo-positioning capabilities are
becoming cheaper and more popular \cite{gps1}. Disclosing their
location information (e.g., via Wi-Fi, Bluetooth, or GPRS), mobile
users can enjoy a variety of location-based services (LBSs). 
One type of such services is a {\em friend-locator} service, which
shows users their friends' locations on a map and/or helps identify
nearby friends. Friend-locator together with other mobile social-networking
services are predicted to become a multi-billion dollar industry over the next
few years \cite{mobsn}. Thus several friend-locator services, like
\textit{iPoki}, \textit{Google Latitude}, and \textit{Fire Eagle}\footnote{
ipoki.com; \; google.com/latitude; \; fireeagle.yahoo.net} 
are now available on the Internet.

In existing friend-locator services, the detection of nearby friends can be done
only manually by a user, e.g., by periodically checking a map on the
mobile device screen. This works only if the user's friends agree to share their
exact locations or at least obfuscated location regions (e.g., city center).
However, LBS users often require some level of privacy and may even feel
threatened \cite{stalk3} if it is not provided. If all user's friends require
complete privacy, they have to disable their location sharing, thus also
preventing the user from finding his or her nearby friends. Consequently, due to
poor location-privacy support, nearby-friend detection is not always
possible in existing friend-locator products.


Nearby friends detection can be enabled to privacy-concerned users utilizing
existing privacy-aware proximity detection methods 
\cite{proxDetTrans,pbsPaper,ffinder,longitudePaper}.
They allow two online users to determine if they are close to each other without
requiring them to disclose their exact locations to a service provider or other
friends. They track all users in the real-time and generate notifications when
friends approach each other.


% Todo: change fig
% either an Euclidean distance between them is smaller than mutually-agreed
% threshold (Fig. \ref{fig:new image}}, 
% 

Generally two users are close to each other if the
\textit{vicinity region} of one user contains the location of other user. The
vicinity region is the area around user's location that defines his ``scope of
interest'' and it can be understood as parameter of spatial range query over his
friends locations. E.g. figure \ref{fig:suppVic}a depicts two users $u_1$ and $u_2$
with their distinctly specified circular-shape vicinities of radii $e_1$ and
$e_2$. Here user $u_2$ is considered to be in $u_1$'s proximity as he falls into
$u_1$ circle with radius $e_1$. On contrary, user $u_1$ is not in proximity of
$u_2$ as he does not fall in $u_2$'s vicinity with radius $e_2$.


Existing proximity detection methods only support static, circular-shaped,
user-location-centered vicinities which radii are typically defined by
\textit{distance thresholds}. Distance threshold is uniquely defined by every
distinct user (Fig. \ref{fig:suppVic}a), or by every pair of friends (Fig.
\ref{fig:suppVic}b). Consequently two friends are in proximity when
Euclidean distance between their locations is no higher than the threshold. 


Supported type of vicinities enable proximity detection only in non-constrained
Euclidean space, e.g. football field with no obstacles, where on
proximity notification users can walk in a straight line to each other. However
if the distance between two users is constrained by the shortest path distance,
that is not always equal to crow-fly distance, then the existing methods are not
applicable. For example, if two users are located on different banks of the
river (Fig. \ref{fig:suppVic}c) such that existing methods classify them being
in proximity, then generated proximity notification might not be very useful for
users. It might be complicated for users to meet each other, because the
distance ($d_2$) of shortest path following the road-network could be much
higher than the crow-fly distance ($d_1$). Moreover, fixed-shape vicinity
based services do not allow users to choose ``areas of interest''. This could be
inconvenient in some cases, e.g. if the user uses the service to find
friends in some bar (Fig. \ref{fig:suppVic}d) while his friends are traveling
along some nearby road with no plans entering the bar. Then the user is flooded
with useless proximity notifications. This scenario is very likely if the user
has many friends and the road is traffic-intensive.

\begin{figure}
       \center
       \begin{tabular}{c c}
			   
\includegraphics[width=0.10\textwidth]{vl/images/vicDemo1.pdf} &
			   
\includegraphics[width=0.15\textwidth]{vl/images/vicDemo2.pdf} \\ 

(a) & (b) \\
			   
\includegraphics[width=0.15\textwidth]{vl/images/rnDemo1.pdf} & 
			   
\includegraphics[width=0.15\textwidth]{vl/images/rnDemo2.pdf} \\
 (c) & (d)
       \end{tabular}
       \caption{User vicinities and proximity detection scenarios}
            \label{fig:suppVic}
\end{figure}

The introduced shortcoming of the scenarios can be eliminated if instead of
static-shape, dynamic-shape vicinities were used. In the first case (Fig.
\ref{fig:suppVic}c), if user $u_1$ at current time moment could select a
vicinity region such that distance between every point of the region and $u_1$
location following the road-network is less than some threshold, then $u_2$'s
detection in $u_1$'s proximity can be avoided. Similarly in the second case
(Fig. \ref{fig:suppVic}d), if user $u_1$ could preselect vicinity region,
matching the bar as it is his ``area of interest'', then only user $u_2$ would
be identified by the system. The challenge is to develop the proximity detection
method that supports both user location privacy and dynamic-shape vicinities. 

To address the challenge, we combine ideas from existing solutions and develop a
client-server location-privacy aware proximity detection service, the \vl. The
proposed solution is based on both spatial cloaking and encryption, where the
central server checks users proximity blindly without knowing any spatial data. 
Users individually are allowed to specify their dynamic vicinities and a
combination of minimum location privacy and service precision. 
Our \vl employs a flexible location-update policy forcing users to update their
location data only when leaving some automatically adjustable regions that
shrink and expand depending on the distance of a user's closest friend.

The paper is organized as follows. The related work is reviewed in
Section \ref{sec:rel_work} followed by our problem setting in
Section \ref{sec:problemdef}. The \vl and experimental results are
presented in Section \ref{sec:contribution} and \ref{sec:performStudy}
respectively.
