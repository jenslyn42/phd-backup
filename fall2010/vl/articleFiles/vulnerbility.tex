\section{Vulnerabilities \& Points of Attack}\label{sec:vulnerability}
We here address a few of the possible vulnerabilities
of the \vl approach. We assume correct behavior of both
the server and clients, and thus exclude attacks where an
attacker may want to modify either server or client to
e.g. trigger a ``spamming'' behavior.
The attackers goal will for each attack be to 
compromise the privacy of the largest possible number of
clients in the \vl system. 


\subsection{Compromised Client}\label{subsec:VulClient}
If an attacker gains control over a client
he will, for each group (see \ref{subsec:grouping})
the client is member of, have the $\Psi$ function 
used to calculate granules at the client.

If the client itself is compromised by an attacker, the $\Psi$ function
is not much help to him, since he cannot do much else than encode granules
sent to the server, but if one imagines that the attacker only temporarily
gains control, then he can use the $\Psi$ function to ``Clone'' the original
client. This problem is however easily made void by changing the $\Psi$ 
function regularly.

By using the \textit{battleship} method 
the attacker can guess the location of other users with same group 
membership as the compromised client\footnote{In the game of \textit{battleships} 
two opponents take turn to guess the location of the others battleships placed at 
secret locations in a grid.}.

One way an attacker may ``play battleship'' in order to find the location of
other users (with same group membership) would be to send a 
false vicinity covering the area he is interested in finding
other users, the attacker will then be notified by the server if any other 
user is within his false vicinity. The attacker then continues to cut the vicinity in half,
doing a binary search until he has found the granules of all users within the larger area
he initially sent his false vicinity for at the start\footnote{The amount of 
granules needed to be searched in the worst case is $\frac{c^{l+1}-1}{c-1}$ 
where $l$ is the number of levels needed to be traversed, and $c$ the number 
of granules each granule at level $l$ is divided into at $l+1$}.

By using groups this attack is already very limited, since each 
client is assumed to have far fewer friends then the overall amount
of users in the \vl system. Furthermore it is worth noticing that 
the attacker can never get an actual location of a user, since all he can get
a matching granule which corresponds to a spacial area and not a point.
There is also a build in limit on the amount of precision the attacker 
can achieve because each users $L_{max}$ is a limit on the precision that
any user will reveal. 
 
If we limit the attackers goal to only focus on a single friend, then 
using the binary search method described will enable the attacker to
track the single friend with the amount of vicinity splits he have to
do in worst case being: $\Theta(Log(\frac{B(cg)}{B(max)}))$ where $B(cg)$ is the
size of attackers current granule and $B(max)$ is the granule size at the maximum
precision attacker can get, either by setting his own $L_{max}$ or reaching the 
friends $L_{max}$.



\subsection{Compromised Server \& Client}\label{subsec:VulCliServ}
If the attacker has gained control over both the server and 
client, he has all info from \ref{subsec:VulClient} as well as 
all users encrypted center and vicinity granules, as well
as their group memberships (see \ref{subsec:grouping}).

The attacker can do the same as in \ref{subsec:VulClient}, 
only now the attacker can skip the \textit{battleship} step and decode
obfuscated location (center granules) of friends directly, making it
actually feasible to track all friends, this however is still
only valid for the groups that the compromised client is member of.

This attack has the same limitations as \ref{subsec:VulClient}, except
that since the attacker now skip the \textit{battleship} stage, it is 
feasible for the attacker to track many users (as long as they are in the
same group as the compromised client)
%
%\subsection{History}\label{subsec:VulHis} \ffh{not relevant, not risk}
%\textbf{info:} All users encrypted center and granules, 
%as well as their group memberships. The attacker collects data over time.
%
%\textbf{can do:} he can try to reason about how the center granule
%moves, by assuming it only moves a cell at a time,
%but this assumption does not hold so attacker can
%really reason about routes unless he has extra info
%about user (e.g. that user is from a specific region of a country,
%then he can maybe brute force possibilities in relation to road networks)


\subsection{Frequency}\label{subsec:VulFreq}
In this attack the server is compromised, and thus the
attacker knows all users encrypted center and granules, 
stored on the server, as well as their group memberships. 

The attacker can compare the frequency of users with same center and 
vicinity granules, the attacker can then see if many users
have the same granules, and reason about the actual location
(e.g. if attacker knows the national soccer team is playing, and he can
see many users suddenly all sharing granules).
The attacker can possibly collect the data over time and maybe make this attack
more efficient by looking for frequency in locations over time, e.g. if 
there is a central place most people must pass during the day (city center/a bridge etc.), 
the attacker can then use historical information to identify which granules 
correspond to this location.

There is a simple solution to thwart the effectiveness of this attack, and that is
to change the $\Psi$ function as some interval, making it impossible for the attacker
to compare granules from different intervals. If we furthermore assume that the server
would not be informed when $\Psi$ function is changed, then this attack becomes void.

