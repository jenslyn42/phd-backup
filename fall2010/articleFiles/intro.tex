\section{Introduction} \label{sec:intro}

\subsection{Prepare for meeting}
formulate problem
 why cache?
\begin{itemize}
\item less computational load
\item faster response time
\item better use of bandwidth \ref{sec:intro}
\item
\end{itemize}

Describe strait forward solution

just store query results in cache and only cosider direct cache hit with a
simpel cache policy like LRU or even FIFO


Come up with some simple solutions

prepare:
\begin{itemize}
\item The definition and problem setting of shortest path caching
\item Some simple methods
\item Running examples to shw how these methods work
\item Identify advantages / disadvantages of these methods
\end{itemize}


\subsection{Problem}

\subsubsection{Definitions and problem setting}
We assume a setting where owners of mobile, positioning enabled, devices want route planning assistance. We assume users prefer online route planning services over offline solutions. 



\subsubsection{methods}


We assume a setting where all users are equipped
with a Mobile Device (MD) able to communicate and
report the users position. All MD’s are online and are
continuesly reporting the users location at predefined
intervals. We use the terms user, mobile device, and
client interchangeable and denote the set of MD’s by
U ⊂ N. We expect a MD to be cabable of visualizing
its current location.
We assume a 2D scenario, where the movements of
users u ∈ U are restricted to a road network G(V, E).
V is the set of vertices, where each vertice v ∈ V
represents either a street intersection or an important
landmark. E is the set of directed edges augmented
by edge length and type. Edges are represented by
a begin/end vertice pair and each edge represents the
smallest unit of a road segment. e ∈ E, each e being
a tuple specifying id, start-/end-vertices, length, and
Road Type (RT) (eid , vs , ve , elength , eRT ). RT is a
hierarchy of the size/type of road i.e. highway, paved,
or dirt road (Sec. 5.2).
The simplest form of trajectory is a collection of tu-
ples (time, longitude, latitude), ordered by the time
attribute, but as we will work on a road network and
in the spatio-temporal domain, such a basic notion of
trajectories is not appropriate. We define T as the set
of trajectories, where each trajectory consist of an id
(tid ), and a sequence of tuples containing an edge and



